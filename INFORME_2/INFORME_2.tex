\documentclass{article}
\usepackage[utf8]{inputenc}
\usepackage{lipsum} % Paquete para generar texto de prueba
\usepackage{graphicx} % Para incluir imágenes
\usepackage{tikz} % Para dibujar figuras

\usepackage{geometry} % Para ajustar el tamaño del texto y los márgenes
% Ajusta el tamaño del texto y los márgenes
\geometry{
    left=30mm, % Margen izquierdo
    top=25mm, % Margen superior
}
\usepackage{float} % Para la opción [H]
\usepackage{enumitem}
\usepackage[spanish]{babel} % Cargar el idioma español

\usepackage{listings}

% Configuración del estilo de las celdas de código
\lstset{
  frame=single, % Agrega un marco alrededor del código
  language=Python, % Lenguaje del código
  basicstyle=\ttfamily\small, % Estilo básico del texto del código
  keywordstyle=\color{blue}, % Estilo de las palabras clave
  commentstyle=\color{green!40!black}, % Estilo de los comentarios
  stringstyle=\color{red}, % Estilo de las cadenas de texto
  showstringspaces=false, % No mostrar espacios en cadenas de texto
  captionpos=b, % Posición de la leyenda (abajo)
}
\lstset{
  literate=%
  {á}{{\'a}}1
  {é}{{\'e}}1
  {í}{{\'i}}1
  {ó}{{\'o}}1
  {ú}{{\'u}}1
  {ñ}{{\~n}}1
}



\usepackage{hyperref}

\title{\textbf{MULTIPLES DISPAROS SINCRONIZADOS}\\[0.5em] \large Investigacion UHPC}
\author{Lukas Wolff Casanova}
\date{\today}
\begin{document}

\maketitle
\hrule

\section{INTRODUCCION}

\noindent Posterior a sincronizar las camaras, se determino que la siguiente tarea era lograr sacar 5000 por camara, lo que da un rango de operacion de 25 segundos a 
200 fps. Para poder lograrlo se postularon varias ideas como listas concatenadas, determinar el tamaño de la lista o usar el buffer de las camaras.

\section{METODOLOGIA}

\noindent Para comenzar se investigo sobre las listas concatenadas o listas con punteros. La diferencia que tienen con una lista normal es que en vez de estar
almacenadas de manera conjunta en la memoria RAM, estas se almacenan los datos de forma particular, junto a un puntero que indica donde esta el siguiente
indice, de esta manera se logra que el proceso de almacenar datos sea mucho mas eficiente en terminos de velocidad y memoria, pero tiene como contra que recorrer 
una lista es mas lento. Tomando a favor la estructura del codigo, donde se recorre la lista una vez que las fotos ya fueron tomadas, se determino que era una
buena herraienta a utilizar.
\\ \\
En primer lugar se crea la clase node, la cual permite crear y recorrer una lista concatenada, la celda mas importante del 
codigo es la siguiente:
\\ \\
\begin{lstlisting}[language=Python]
  # Método para recorrer la lista de nodos
  def traverse(self):
      current_node = self.head
      
      while current_node:

          img = current_node.data.GetArray()
          img = cv2.equalizeHist(img)
          camara = current_node.data.GetCameraContext()
          tiempo = current_node.data.GetTimeStamp()

          # Carpeta donde se almacenarán las imágenes
          folder_path = f'FOTOS/{RPM}/{camara}'
          print(folder_path)

          # Crear la carpeta si no existe
          if not os.path.exists(folder_path):
              os.makedirs(folder_path)

          # Nombre del archivo de la imagen
          file_name = f'{tiempo}.png'

          # Ruta completa del archivo de la imagen
          file_path = os.path.join(folder_path, file_name)

          # Crear una imagen PIL a partir de la matriz de la imagen
          image = Image.fromarray(img)

          # Guardar la imagen como PNG
          image.save(file_path)

          print(f"Imagen guardada: {file_path}")
          current_node = current_node.next
\end{lstlisting}

Como se puede ver, todo el proceso que se realizaba anteriormente al recorrer la lista imagenes, ahora
se realiza dentro de la classe node, la cual es llamada al final del codigo:

Posteriormente, se setean los ajustes tanto de camaras como fotos:




























\noindent Enlace a GitHub con toda la información: \href{https://github.com/LukasWolff2002/SINCRONIZACION_CAMARAS_BASLER}{GIT-HUB-SINCRONIZACION-CAMARAS}.
\\ \\
\noindent Enlace a GitHub con codigo base: : \href{https://github.com/basler/pypylon/blob/master/samples/grabmultiplecameras.py}{GIT-HUB-CODIGO-BASE}.


\end{document}
